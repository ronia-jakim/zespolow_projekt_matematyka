\section{Wstęp}

\subsection{Rozmaitość Riemannowska i tensor metryczny}

Czasoprzestrzeń w ogólnej teorii względności jest $4$ wymiarową rozmaitością, której trzy współrzędne oznaczają położenie w przestrzeni, a czwarta współrzędna informuje nas o czasie. Z matematycznego punktu widzenia, rozmaitość to dowolna przestrzeń topologiczna $M$ taka, że dla każdego punktu $p\in M$ możemy znaleźć otwarty zbiór $p\in U_p\subseteq M$ który jest homeomorficzny z pewnym podzbiorem $\R^n$. To znaczy, że rozmaitości są lokalnie Euklidesowe. Taka definicja czasoprzestrzeni jest jednak bardzo ogólna, więc wiele modeli sięga po nieco bardziej restrykcyjną definicję rozmaitości.
=======
Pojęcie gładkiej rozmaitości pozwala na opisywanie konstrukcji geometrycznych poprzez lokalne sprowadzenie ich za pomocą gładkich map do podzbiorów $\R^n$ (lub $\C^n$). To znaczy, że mając pewną abstrakcyjną gładką rozmaitość $M$, dostajemy jednocześnie dostęp do rodziny par $\{(U_\alpha,\phi_\alpha\}$, gdzie zbiory $\{U_\alpha\}$ tworzą otwarte pokrycie $M$, a 
$$\phi_\alpha:U_\alpha\to V_\alpha\subseteq \R^n$$
jest gładką, ciągła funkcją między $U_\alpha$ a podzbiorem $V_\alpha\subseteq\R^n$. Takie funkcje nazywamy dyfeomorfizmami. Warto tutaj zaznaczyć, że liczba $n$ jest stała dla całej rozmaitości i dzięki temu pojęcie wymiaru rozmaitości $M$, $\dim M=n$, jest dobrze określone.

Biorąc dowolny punkt $p\in M$ możemy dla niego znaleźć pewną mapę $(U_\alpha,\phi_\alpha)$ taką, że $p\in U_\alpha$. Ponieważ $\phi_\alpha(U_\alpha)$ jest pewnym obiektem w $\R^n$, to możemy mówić o wektorach do niego stycznych. Nasuwa się więc pytanie, czy możemy pojęcie styczności wektora do obiektu uogólnić do styczności wektorów do rozmaitości różniczkowalnych. Idąc tym tokiem myślenia bardzo szybko trafiamy na pojęcie przestrzeni stycznej do $M$ w punkcie $p$: $T_pM$. 

Bardzo ciekawą własnością przestrzeni stycznych jest ich liniowość, to jest obiekt $T_pM$ jest przestrzenią liniową wymiaru $n$. Wybierając mapę $(U, \phi)$ wokół punktu $p\in M$ dostajemy od razu bazę na $T_pM$, której elementy zwyczajowo oznaczamy
$$\{ \frac{\partial}{\partial\phi_1} , ... , \frac{\partial}{\partial\phi_n} \}.$$








Do dowolnej gładkiej rozmaitości $M$ wymiaru $n$ możemy dołączyć wiązkę styczną $TM$, która dla dowolnego punktu $p\in M$ zawiera przestrzeń liniową $T_pM$ nazywaną przestrzenią styczną do $M$ w punkcie $p$. 
>>>>>>> fbb51673c34c4bb7061bcf9ce358e1cc1893cda7

\begin{definition}\label{def:rozmaitosci rozniczkowalna}
  \buff{Rozmaitość różniczkowalna} wymiaru $\R^n$ to para $(M,\set{A})$, gdzie $M$ jest rozmaitością, a $\set{A}$ jest maksymalnym atlasem gładkim. To znaczy, $\set{A}$ jest największą, co do zawierania, rodziną map $(U_\alpha,\phi_\alpha)$ taką, że 
  \begin{itemize}
    \item zbiory mapowe $U_\alpha$ są otwarte i pokrywają $M$ [$\bigcup U_{\alpha}=M$],
    \item dla każdego $\alpha$ odwzorowanie $\phi_\alpha:U\to \overline{U}\subseteq\R^n$ jest homeomorfizmem, a liczba $n$ jest jedyna dla $M$,
    \item dowolne dwie mapy $(U_\alpha,\phi_\alpha), (U_\beta,\phi_\beta)\in\set{A}$ są gładko zgodne, tzn. 
      \begin{itemize}
        \item ich dziedziny nie pokrywają się, $U_\alpha\cap U_\beta=\emptyset$, lub
        \item jeśli $U_\alpha\cap U_\beta\neq \emptyset$ to mapy przejścia, $\phi_\alpha\phi_\beta^{-1}$ i $\phi_\beta\phi_\alpha^{-1}$, są gładkimi odwzorowaniami pomiędzy podzbiorami $\R^n$.
      \end{itemize}
  \end{itemize}
\end{definition}

Często wybierając dowolny punkt $p\in M$ chcemy rozważyć jedną z map $(U_\alpha,\phi_\alpha)$ taką, że $p\in U_\alpha$. Taką dowolną mapę zawierającą $p\in M$ będziemy oznaczać jako $(U_p,\phi_p)$. Dodatkowo, możemy wymagać od takiego zbioru $U_p$, by jego obraz przez $\phi_p$ był kulą wokół środka układu współrzędnych, tzn.
$$\phi_p(U_p)=\overline{U_p}\{(x_1,...,x_n)\;:\;\|(x_1,...,x_n)\| < r\}=B_r(0)\subseteq\R^n.$$
Wówczas $\phi_p(p)=(0,...,0)\in\overline{U_p}\subseteq\R^n$ \cite{leeSmoothManifolds}.

Warto zauważyć, że przestrzenie $\R^k$ spełniają definicję rozmaitości różniczkowalnych, np. z atlasami: 
$$\set{A}=\{(\R^k,Id_{\R^k})\}$$
$$\set{A}'=\{ (B_1(x), Id_{B_1(x)}) \;:\;x\in\R^k\}$$

\begin{definition}\label{def:funkcja gładka}
  Niech $(M, \set{A})$ i $(N,\set{B})$ będą rozmaitościami gładkimi. Funkcja $f:M\to N$ taka, że dla dowolnych punktów $p\in M$ i $f(p)=q\in N$ istnieją mapy $(U_p,\phi_p)$ wokół $p$ oraz $(V_q,\psi_q)$ wokół $q$ takie, że reprezentacja $f$ 
  $$\psi_q\circ f\circ\phi_p^{-1}:\phi_p(U_p)\to \psi_q(V_q)$$
  w tych mapach jest funkcją gładką między $\phi_p(U_p)$ i $\psi_q(V_q)$:
\end{definition}

Do dowolnej gładkiej rozmaitości $M$ wymiaru $n$ możemy dołączyć wiązkę styczną $TM=\bigsqcup_{p\in M}T_pM$, gdzie $T_pM$ jest przestrzenią styczną do $M$ w punkcie $p$. Ważną własnością przestrzeni stycznych jest ich liniowość.

Istnieje więc w matematyce sposób na rozważanie abstrakcyjnych przestrzeni przez pryzmat dobrze zbadanych $\R^n$ oraz dopisanie do nich struktury przestrzeni liniowej dzięki przestrzeniom stycznym. Idąc dalej, możemy zastanowić się jakie inne właściwości przestrzeni euklidesowych możemy uogólnić na abstrakcyjne przestrzenie $T_pM$.

Rozmaitość Riemanna to gładka, rzeczywista rozmaitość $M$ z rodziną dwuliniowych funkcji 
$$g_p:T_pM\times T_pM\to \R$$ 
zdefiniowanych w każdym punkcie $p\in M$. Każda taka funkcja $g_p$ jest dodatnio określonym iloczynem wewnętrznym na $T_pM$, a więc pociąga za sobą normę 
$$\|v\|_p=\sqrt{g_p(v, v)}.$$
W ten sposób możemy na $TM$ określić funkcję $g$, która dowolnym dwóm wektorom $X_p,Y_p$ zaczepionym w tym samym punkcie $p\in M$ przypisuje ich odpowiednik iloczynu skalarnego $g_p(X_p,Y_p)$. Tak określony funkcjonał nazywamy tensorem metrycznym, lub w skrócie metryką, na rozmaitości $M$.
%Funkcja $g$ określona na $TM$, która wektorowi zaczepionemu w punkcie $p$ przypisuje jego normę $\|\cdot\|_p$ jest nazywana tensorem metrycznym, lub w skrócie metryką, na rozmaitości $M$.

Dowolna mapa $(U,\phi)$ na $n$-wymiarowej Riemannowskiej rozmaitości $M$ zawierająca punkt $p\in M$ pociąga za sobą bazę przestrzeni $T_pM$
$$\left\{ \frac{\partial}{\partial\phi_1} , ... , \frac{\partial}{\partial\phi_n} \right\}$$
w takim razie funkcjonał $g_p$ zapisuje się macierzą o wymiarze $n\times n$. Wyrazami takiej macierzy są wartości $g_p$ na kolejnych parach wektorów bazowych.

Mając bazę dualną do $\{\frac{\partial}{\partial\phi_i}\}$ możemy zapisać tensor metryczny za pomocą prawdziwego tensora
$$g=\sum_{i,j\leq n}g_{i,j}d\phi^i\otimes d\phi^j$$
gdzie $g_{i,j}$ to wyrazy macierzy wspomnianej wyżej, a $d\phi^i$ to elementy bazy dualnej do $\frac{\partial}{\partial\phi_i}$


\subsection{Czarne dziury Schwarzschild'a}

Jednym z najprostszych, a przez to najczęściej używanych, sposobów opisu przestrzeni wokół czarnej dziury jest modelowanie tej osobliwości jako sferycznie symetrycznego obiektu o pewnej masie, pozbawionego ładunku elektrycznego i przyśpieszenia kątowego. Tak zdefiniowane czarne dziury nazywamy czarnymi dziurami Schwarzschild'a, na pamiątkę niemieckiego fizyka który jako pierwszy znalazł dokładne rozwiązanie równania Einsteina. W języku matematyki, czarna dziura którą zajmiemy się w tej pracy jest modelowana przez rozmaitość
$$\R\times(0, +\infty)\times S^2$$
czyli w tym przypadku współrzędne będą oznaczać kolejno czas, odległość od środka masy i położenie na $S^2$ zapisywane przy pomocy dwóch kątów.

Wspomniane rozwiązanie nazywa się metryką Schwarzschild'a i zapisuje się je jako
$$g=-c^2d\tau^2=- \left( 1-\frac{r_s}{r}  \right) c^2dt^2 + \left(1-\frac{r_s}{r} \right)^{-1}dr^2 +r^2(d\theta^2+\sin^2(\theta)d\phi^2) $$
lub w formie macierzy
$$g_{\mu\nu}=\begin{bmatrix}
  -\left[1-\frac{r_s}{r}\right] & 0 & 0 & 0\\ 
  0 & \left[1-\frac{r_s}{r}\right]^{-1} & 0 & 0\\ 
  0 & 0 & r^2 & 0\\ 
  0 & 0 & 0 & r^2\sin^2(\theta)
\end{bmatrix},$$ 
gdzie $\tau$ to czas własny (mierzony przez zegar poruszający się razem z cząsteczką testową), $t$ to współrzędna czasu (mierzona przez zegar położony nieskończenie daleko od czarnej dziury), a $r,\theta,\phi$ to współrzędne sferyczne wokół środka czarnej dziury ($\theta$ to kąt między wektorem opisującym położenie a północnym biegunem przestrzeni) \cite{notatkiUoCSD}.

%W każdym punkcie $(t, r, \theta, \phi)$ rozmaitości opisującej czasoprzestrzeń wokół badanej czarnej dziury możemy rozważać przestrzeń do niej styczną. Bazą takiej przestrzeni będą wówczas wektory $\frac{\partial}{\partial t}, \frac{\partial}{\partial r}, \frac{\partial}{\partial \theta}$ i $\frac{\partial}{\partial \phi}$ których długości to
%$$\left\|\frac{\partial}{\partial\mu}\right\|_g=\sqrt{g_{\mu\mu}}$$
%czyli pierwiastki odpowiednich wyrazów macierzy $g_{\mu\nu}$. Zauważmy, że długość żadnego z wektorów nie jest zależna od czasu $t$ ani od szerokości geograficznej.




